\chapter{Herramientas básicas}

\section{Tipos de modelos}

Los modelos discretos (por ejemplo SI, SIR y SIS) usan las etiquetas Susceptible, Infectado y Recuperado. Los nombres suelen hacer referencias al flujo que se sigue para pasar entre las etiquetas. Así, por ejemplo un modelo SI pasa de susceptible a infectado, uno SIR de susceptible, infectado y recuperado y SIS alterna entre susceptible e infectado.

En estos modelos se hacen dos suposiciones:
\begin{enumerate}
\item La población se mezcla de manera homogénea, es decir, todos los individuos tienen la misma probabilidad de contraer la enfermedad.
\item El total de la población es constante.
\end{enumerate}

\subsection{Modelo SI}
Es el modelo más simple de todos, los individuos nacen siendo susceptibles a una enfermedad, y una vez infectados no hay tratamiento y permanecen infectados el resto de su vida.
Un ejemplo de una enfermedad que pueda modelarse usando SI es el herpes.

\begin{equation}
\label{eqn: SI_S}
S_{n+1}=S_n\left( 1-\frac{\alpha\Delta t}{N}I_n\right)
\end{equation}

\begin{equation}
\label{eqn: SI_I}
I_{n+1}=I_n\left( 1+\frac{\alpha\Delta t}{N}S_n\right)
\end{equation}

Con condiciones iniciales $S_0>0$, $I_0>0$ y $S_0+I_0=N$.

En estas ecuaciones $\alpha$ es la tasa de contacto, esto es, el número medio de individuos con los que un infectado tiene suficiente contacto para contagiarlo en un intervalo de tiempo. Por tanto, $S_n$ representa el número de individuos susceptibles en el tiempo $n\Delta t$.

Ahora, imponemos las suposiciones descritas anteriormente para estos modelos. Empezamos por la segunda: La población total se mantiene constante, que es trivial que se cumpla siempre, ya que sumando el sistema de ecuaciones el resultado es $N$ y asumimos que las soluciones son siempre positivas (las soluciones negativas no tienen sentido).
Para imponer que las ecuaciones tienen soluciones positivas:
En el caso de la $S_n$ una condición necesaria y suficiente es $\alpha\Delta t \leq 1$. 

Buscamos ahora ver cuál es el comportamiento del sistema, si calculamos los puntos de equilibrio del sistema, para lo que resolvemos:

$$
\begin{cases}
S^*=S^*\left( 1-\frac{\alpha\Delta t}{N}I^*\right) \\
I^*=I^*\left( 1+\frac{\alpha\Delta t}{N}S^*\right) \\
S^*+I^*=N
\end{cases}
$$

Los únicos puntos de equilibrio posibles son: $S^*=0, I^*=N$ y $S^*=N, I^*=0$, y como sabemos que tenemos condiciones iniciales positivas y $S_n$ es monótonamente decreciente e $I_n$ es monótonamente creciente, entonces debe converger a $S^*=0, I^*=N$.

\textcolor{red}{Aquí en el artículo (Allen, página 3) \cite{allenDiscretetimeSISIR1994} lo hace de otra manera, sin sustituir el S+I=N en el sistema y eso lo hace luego en forma alternativa, pero dice que eso añade restricciones por obtenerse la ecuación logística y no lo entiendo bien. ¿Qué hace al prinicpio si no sustituye del sistema? ¿De dónde salen las restricciones extra de la ecuación logística?}

Expresando $\alpha$ como una tasa podemos obtener las ecuaciones diferenciales análogas de la sigueinte manera:

$$\frac{S_{n+1} - S_n}{\Delta t} \approx \frac{dS}{dt}$$

luegosu análoga continua es:

\begin{equation}
\frac{dS}{dt} = -\frac{\alpha}{N}SI
\end{equation}


\begin{equation}
\frac{dI}{dt} = \frac{\alpha}{N}SI
\end{equation}

con condiciones iniciales $S(0)+I(0)=N$.

De manera análoga al caso discreto, podemos comprobar que este sistema converge a $S^*=0, I^*=N$ y, por tanto, tiene el mismo comportamiento que el caso discreto.
\textcolor{red}{Aquí llega al final de la página 3 del pdf del artículo de Allen\cite{allenDiscretetimeSISIR1994}, pero no entiendo bien como calcula el punto de equilibrio al que converge el sistema continuo}

\textcolor{red}{La página 4 del artículo sigue haciendo cosas que no entiendo y me pierdo bastante}


\subsection{Modelo SIS}
Es similar al SI, pero tras infectarse los individuos vuelven a ser susceptibles.
Por ejemplo, los resfriados pueden modelarse usando SIS.

\subsection{Modelo SIR}
Comienza como el SI, pero tras infectarse los individuos pasan a un estado Recuperado, en el cuál no pueden infectarse ni infectar a otros.
Por ejemplo, la varicela. 


\section{Cosas del articulo de rsos que no sé que nombre ponerles}

\subsection{Introducción}

Estimar la tasa de transmisión media es uno de los aspectos más cruciales en epidemiología. Esta tasa condiciona la fase de la epidemia (incluso si va a extinguirse). Es combinación de tres factores:

\begin{enumerate}
\item Coeficiente de virulencia: Relacionado con el agente infeccioso.
\item Coeficiente de susceptibilidad: Relacionado con el anfitrión.
\item Número de contactos por unidad de tiempo entre individuos.
\end{enumerate}

Los dos primeros factores se tienen en cuenta a la vez en la probabilidad de transmisión.

Todos los factores pueden cambiar con el tiempo, el primero debido a mutaciones del virus y los dos últimos por medidas de contención. Por tanto, observar el decrecimiento de la transmisión media en una enfermedad es una buena forma de comprobar la efectividad de las medidas de contención.

Consideramos un modelo SI con el objetivo de compararlo con los datos obtenidos en la pandemia hasta el momento y así tratar de prdecir su comportamiento en el futuro.

El modelo SI es el siguiente:

\begin{equation}
\label{eqn: SI_S_cont}
S'(t) = -\tau (t)S(t)I(t)
\end{equation}

\begin{equation}
\label{eqn: SI_I_cont}
I'(t) = \tau (t)S(t)I(t) -vI(t)
\end{equation}

donde $S(t)$ es el número de individuos susceptibles , $I(t)$ el número de individuos infectados en el tiempo $t$ y $\tau$ la tasa de transmisión, que combina el número de contactos por unidad de tiempo y la probabilidad de transmisión. Además, notemos que $v$ es constante, donde $1/v$ es la duración media del período de infección, y $vI(t)$ el flujo de individuos recuperados o muertos \textcolor{red}{¿Por que ese es el flujo de recuperados o muertos?}. Consideramos condiciones iniiciales:

$$S(t_0)=S_0>0, \: I(t_0)=I_0>0$$

Ahora, consideramos que al final del período infeccioso nos han reportado una fracción del total de casos, en este caso $f\in (0,1]$. Sea $CR(t)$ el número total (acumulado) de casos reportados. Entonces:

\begin{equation}
\label{eqn: acumulada}
CR(t) = CR_0 + vfCI(t) \; \forall t \geq t_0
\end{equation}

\textcolor{red}{En esta ecuación no especifica que $CR_0$ sea constante, pero asumo que es un valor inicial, porque tendría sentido. También tengo duda en ¿por qué $vfCI(t)$ va multiplicado por $v$?}

donde

$$CI(t) = \int_{t_0}^t I(\sigma ) d\sigma $$

Asumimos conocidos $S_0 > 0$, $1/v>0$, $f\in (0,1]$. Por tanto, queremos averiguar $I_0$, $\tau (t)$

\subsection{Aproximando $I_0$ y $\tau (t_0)$}
Ahora, procedemos a intentar aproximar $I_0$ y $\tau (t_0)$:

Al comienzo de la pandemia podemos asumir que $S(t)$ y $\tau (t)$ son constantes e iguales a $S_0$ y $\tau_0 = \tau (t_0)$ respectivamente. Así, sustituyendo estos valores en la ecuación \ref{eqn: SI_I_cont} obtenemos:

$$I'(t) = (\tau_0 S_0 -v) I(t)$$

Resolviendo la ecuación diferencial llegamos a:

$$I(t) = I_0\exp{((\tau_0 S_0-v)(t-t_0))}$$

Sustituyendo en \ref{eqn: acumulada}:

$$CR(t) = CR_0 + vfI_0\frac{\mathrm{e}^{•(t-t_0)} -1}{\tau_0 S_0-v}$$

Así, hemos obtenido un primer modelo para los casos acumulados al principio de la pandemia.

Reescribimos la ecuación como:

\begin{equation}
\label{eqn: acumulada_modelo}
CR(t) = \chi_1 \mathrm{e}^{\chi_2 t} -\chi_3
\end{equation}

Estimamos $\chi_3$ usando los datos de la epidemia obtenidos, y el mejor ajuste para los datos es $\chi_3=0$

Ahora, usando \ref{eqn: acumulada} y \ref{eqn: acumulada_modelo} tenemos:

\begin{equation}
I_0=\frac{\chi_1\chi_2\mathrm{e}^{\chi_2 t_0}}{vf}
\end{equation}

Y, como de reescribir sabemos que $\chi_2 = \tau_0 S_0-v$ entonces

\begin{equation}
\tau_0 = \frac{\chi_2+v}{S_0}
\end{equation}

Si suponemos que $\tau (t) = \tau_0$ tenemos que el modelo queda:

\begin{equation}
S'(t) = -\tau_0S(t)I(t)
\end{equation}

\begin{equation}
I'(t) = \tau_0S(t)I(t) -vI(t)
\end{equation}

Usando la ecuación de $S(t)$ y resolviéndola obtenemos:

$$S(t) = S_0\exp{(-\tau_0 \int_{t_0}^t I(\sigma ) d\sigma)} = S_0\exp{(-\tau_0 CI(t))}$$

Ahora, sustituyendo esta expresión en la ecuación de $I(t)$ del modelo y usando $CI'(t)=I(t)$:

$$I'(t) = S_0\exp{(-\tau_0 CI(t))}\tau_0 CI'(t)-vI(t)$$

Finalmente, integrando entre $t_0$ y $t$ tenemos que:

$$I(t)=CI'(t)=I_0+S_0(1-\exp{(-\tau_0 CI(t)}))-vCI(t)$$

Observamos entonces que el número total de infectados es monótono creciente (¡pero no el número de infectados!). 

\begin{theorem}
Sea $t>t_0$ fijo. El número de infectados acumulados es estrictamente creciente \textcolor{red}{¿Dónde hemos probado que sea estricto? Creciente si, pero estricto?} respecto a las siguiente cantidades:
\begin{itemize}
\item $I_0>0$ Número inicial de infectados
\item $S_0>0$ Número inicial de individuos susceptibles.
\item $\tau>0$ Tasa de transmisión
\item $1/v$ Tiempo medio de la infección.
\end{itemize}
\end{theorem}

\textcolor{red}{Ahora en el artículo de \cite{demongeotSIEpidemicModel} menciona una cosa de errores que me da miedo}

\textcolor{red}{La figura 2 tampoco la entiendo mucho :(}







