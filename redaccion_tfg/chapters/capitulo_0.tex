%\ctparttext{\color{black}\begin{center}
%		Esta es una descripción de la parte de informática.
%\end{center}}

%\part{Parte de informática}
\chapter{Herramientas básicas}

En esta sección se presentarán algunos conceptos y herramientas básicas que nos resultarán útiles a lo largo del desarrollo en los siguientes capítulos.

\section{Ecuaciones en diferencias}

\begin{definition}
Se define una \textbf{ecuación en diferencias de orden $k$} como una ecuación en la que intervienen un número fijo de términos consecutivos de una sucesión.

\begin{equation}
\label{def_ec_diferencias}
F(n,X_n, X_{n+1}, \cdots , X_{n+k}) = 0 \forall n\in\mathbb{N}
\end{equation}


donde $F$ es una función de $k+2$ variables definida en $F:\mathbb{N}\times I^{k+1}\rightarrow \mathbb{R}$, donde $I$ es un intervalo con al menos dos puntos.
\end{definition}

\begin{definition}
Una ecuación en diferencias de orden $k$ se dice que está en \textbf{forma normal} si está expresada de la forma:

\begin{equation}
\label{def_ec_forma_normal}
X_{n+k} = \Phi (n, X_n, X_{n+1}\cdots , X_{n+k-1})
\end{equation}


donde $\Phi$ es una función dada definida en $\Phi :\mathbb{N}\times I^{k}\rightarrow I$.
\end{definition}

\begin{definition}
Una \textbf{solución de una ecuación en diferencias} de orden $k$ es una sucesión $X_n$ definida explícitamente de la forma:

$$X_n = f(n)$$

donde $f: \mathbb{N} \rightarrow I$ es una función que cumple (\ref{def_ec_diferencias}) al sustituir $X_n$ por $f(n)$ para cualquier $n\in\mathbb{N}$.

El conjunto de todas las soluciones de (\ref{def_ec_diferencias}) se llama \textbf{solución general} de la ecuación.
\end{definition}

\begin{theorem}[Existencia y unicidad]
Sea $\Phi$ una función como en (\ref{def_ec_forma_normal}), entonces la ecuación en diferencias en forma normal (\ref{def_ec_forma_normal}) tiene solución.
Para cada k-tupla $(\alpha_0, \alpha_1, \cdots ,\alpha_{k-1})\in I^{k}$ el problema anterior en forma normal con condiciones iniciales las indicadas tiene una única solución.

\begin{equation}
\begin{cases}
X_{n+k} = \Phi (n, X_n, X_{n+1}\cdots , X_{n+k-1}) \\
X_0 = \alpha_0, X_1=\alpha_1, \cdots X_{k-1}=\alpha_{k-1}
\end {cases}
\end{equation}

\end{theorem}
\begin{proof}
Sustituyendo las condiciones iniciales en (\ref{def_ec_forma_normal}) con $k=0$ obtenemos el valor exacto para $X_k$, al que llamamos $\alpha_k$. Por definición de $\Phi$ tenemos que $(\alpha_1, \alpha_2, \cdots ,\alpha_{k})\in I^{k}$, luego sustituyendo de nuevo en la misma ecuación por las condiciones iniciales

$$X_1 = \alpha_1, X_2=\alpha_2, \cdots X_{k}=\alpha_{k}$$

obtenemos $X_{k+1}$. Repitiendo este proceso se crea una sucesión $X_n$ única, que es solución de (\ref{def_ec_forma_normal}).
\end{proof}



\subsection{Subseccion}

aqui van cosas