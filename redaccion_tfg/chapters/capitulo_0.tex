%\ctparttext{\color{black}\begin{center}
%		Esta es una descripción de la parte de informática.
%\end{center}}

%\part{Parte de informática}
\chapter{Herramientas básicas}

En esta sección se presentarán algunos conceptos y herramientas básicas que nos resultarán útiles a lo largo de los siguientes capítulos.

\section{Ecuaciones en diferencias}

Comenzamos presentando alguna teoría básica que puede consultarse en \cite{salinelliDiscreteDynamicalModels2014}

\begin{definition}
Se define una \textbf{ecuación en diferencias de orden $k$} como una ecuación en la que intervienen un número fijo de términos consecutivos de una sucesión.

\begin{equation}
\label{def_ec_diferencias}
F(n,X_n, X_{n+1}, \cdots , X_{n+k}) = 0 \quad \forall n\in\mathbb{N}
\end{equation}


donde $F$ es una función de $k+2$ variables definida en $F:\mathbb{N}\times I^{k+1}\rightarrow \mathbb{R}$, con $I$ un intervalo con al menos dos puntos y donde la sucesión $\{X_n\}_{n\geq 0}$ es la incógnita.
\end{definition}

\begin{definition}
Una ecuación en diferencias de orden $k$ se dice que está en \textbf{forma normal} si está expresada de la forma:

\begin{equation}
\label{def_ec_forma_normal}
X_{n+k} = \Phi (n, X_n, X_{n+1}\cdots , X_{n+k-1})
\end{equation}


donde $\Phi$ es una función dada definida en $\Phi :\mathbb{N}\times I^{k}\rightarrow I$.
\end{definition}

\begin{definition}
Una \textbf{solución de una ecuación en diferencias} de orden $k$ es una sucesión $\{X_n\}_{n\geq 0}$ de números reales definida explícitamente de la forma:

$$X_n = f(n),\quad n\geq 0$$

donde $f: \mathbb{N} \rightarrow I$ es una función que cumple (\ref{def_ec_diferencias}) al sustituir $\{X_n\}_{n\geq 0}$ por $f(n)$ para cualquier $n\in\mathbb{N}$.

El conjunto de todas las soluciones de (\ref{def_ec_diferencias}) se llama \textbf{solución general} de la ecuación.
\end{definition}

\begin{theorem}[Existencia y unicidad]
Sea $\Phi$ una función de la forma presentada en (\ref{def_ec_forma_normal}). Entonces la ecuación en diferencias en forma normal (\ref{def_ec_forma_normal}) tiene solución.
Para cada k-upla $(\alpha_0, \alpha_1, \cdots ,\alpha_{k-1})\in I^{k}$ el problema anterior en forma normal con condiciones iniciales dadas tiene una única solución.

\begin{equation}
\begin{aligned}
X_{n+k} = \Phi (n, X_n, X_{n+1}\cdots , X_{n+k-1}) \\
X_0 = \alpha_0, X_1=\alpha_1, \cdots X_{k-1}=\alpha_{k-1}.
\end {aligned}
\end{equation}

\end{theorem}
\begin{proof}
Sustituyendo las condiciones iniciales en (\ref{def_ec_forma_normal}) con $k=0$ obtenemos el valor exacto para $X_k$, al que llamamos $\alpha_k$. Por definición de $\Phi$, $(\alpha_1, \alpha_2, \cdots ,\alpha_{k})\in I^{k}$, luego sustituyendo de nuevo en la misma ecuación por las condiciones iniciales

$$X_1 = \alpha_1, X_2=\alpha_2, \cdots X_{k}=\alpha_{k}$$

obtenemos $X_{k+1}$. Repitiendo este proceso se crea una sucesión $\{X_n\}_{n\geq 0}$ única, que es solución de (\ref{def_ec_forma_normal}).
\end{proof}


\subsection{Ecuación logística}

\begin{definition}
La ecuación logística discreta tiene la siguiente forma:

$$X_{n+1} = \mu X_n(1-X_n),\quad \mu > 0$$
\end{definition}

\begin{proposition}
En la ecuación logística, se tiene que su solución converge si $\mu < 3$, oscila si $3 < \mu \lesssim 3.57$ y se produce caos si $\mu \gtrsim 3.57$. 
\end{proposition}

\textcolor{red}{TODO explicar toda la teoría de la ecuación logística}

\subsection{Sistemas de ecuaciones en diferencias}

\begin{definition}
Se define un \textbf{sistema de ecuaciones en diferencias} como un conjunto de ecuaciones en diferencias a resolver.

\begin{equation}
\begin{cases}
X_{n+k} = \Phi_1(k, X_n, \cdots , X_{n+k-1}, Y_n, \cdots, Y_{n+k-1}) \\
Y_{n+k} = \Phi_2(k, X_n, \cdots , X_{n+k-1}, Y_n, \cdots, Y_{n+k-1}) \\
\end{cases}
\end{equation}

Renombrando, podemos expresar el conjunto de sucesiones como:

\begin{equation}
X_n = (X_{1_n}, X_{2_n}, \cdots , X_{m_n})
\end{equation}

luego el sistema de ecuaciones resultaría:

\begin{equation}
\label{def_sist_ec}
X_{n+k} = \Phi (n, X_n, \cdots , X_{n+k-1})
\end{equation}


donde $\Phi : D \rightarrow D$,, con $D$ siendo el dominio de la función $\Phi$, y por tanto $D\subset \mathbb{N}\times\mathbb{R}^k$ y con $\Phi = (\Phi_1, \cdots \Phi_m)$.

\end{definition}

\begin{definition}
Una \textbf{solución} de un sistema de ecuaciones en diferencias son las soluciones explícitas de cada ecuación del sistema, de forma que se satisfagan todas ellas; es decir, encontrar una solución explícita para $X_n$ en (\ref{def_sist_ec}).
\end{definition}

\subsection{Puntos de equilibrio}

\begin{definition}
Sea $D\subset \mathbb{R}^m$ y $\Phi :D\rightarrow D$ una función continua. Entonces un vector $\alpha \in \mathbb{R}^m$ se dice que es un \textbf{punto de equilibrio} de un sistema de ecuaciones en diferencias de la forma (\ref{def_sist_ec}) si

$$\alpha = \Phi (\alpha),\quad \alpha \in D.$$

Son soluciones independientes del tiempo.
\end{definition}

\begin{proposition}
Si tomamos como valor inicial un punto de equilibrio, llamemoslo $\alpha$, de un sistema de ecuaciones en diferencias, entonces la solución explícita del sistema es constante y es:
$$X_n = \alpha \quad \forall n\in\mathbb{N}.$$
\end{proposition}

\begin{definition}
Un punto de equilibrio $\alpha$ de un sistema de ecuaciones en diferencias como (\ref{def_sist_ec}) se dice que es un \textbf{atractor global} si para cualquier $X_0\in D$ se verifica
$$\displaystyle\lim_{n\to \infty} X_n = \alpha.$$
\end{definition}

\begin{definition}
Un punto de equilibrio $\alpha$ de un sistema de ecuaciones en diferencias como (\ref{def_sist_ec}) se dice que es un \textbf{atractor local} si existe $\eta>0$ tal que para cualquier $X_0\in D\cap \text{]}\alpha -\eta , \alpha + \eta \text{[}$ se verifica
$$\displaystyle\lim_{n\to \infty} X_n = \alpha.$$

\end{definition}