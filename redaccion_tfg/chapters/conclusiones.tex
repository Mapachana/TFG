%\ctparttext{\color{black}\begin{center}
%		Esta es una descripción de la parte de informática.
%\end{center}}

%\part{Parte de informática}
\chapter{Conclusiones y vías futuras}

\section{Conclusiones}

Los modelos epidemiológicos, ya sean discretos o continuos, son un método intuitivo de analizar comportamientos de diversas enfermedades. Aunque los modelos presentados en este proyecto son modelos relativamente sencillos, siguen siendo una herramienta muy útil en el campo de la epidemiología.

Estos modelos presentan muchas limitaciones, ya que se usan suposiciones que rara vez pasan en casos reales, porque la población de un lugar nunca es constante, hay gente entrando y saliendo, y no es cierto que todos los individuos interactúen con todos los demás individuos de la población, pues una persona no tiene relación con todas las personas del resto de España. Aún con estas restricciones que en la realidad no se cumplen, hemos comprobado que los modelos ajustan relativamente bien los datos reales. También se supone que todo individuo infectado es siempre igual de contagioso, mientras que la carga viral y por tanto cómo de contagioso resulta a otras personas varía en el tiempo.

Sumado a estos problemas, tenemos que tener en cuenta también el ruido y la incompletitud que tienen los datos reales. Dado que realizar experimentos en epidemiología es muy difícil, se suele trabajar con datos reales de epidemias, lo que supone muchos problemas. Muchas veces estos datos están incompletos, ya que algunos días pueden no reportarse los casos nuevos infectados, o los reportes de unos lugares con otros pueden no ser consistentes. Por ejemplo durante la epidemia del covid, cada comunidad autónoma tenía un sistema propio para realizar los reportes de datos, y este además fue cambiando a lo largo del desarrollo de la pandemia, produciendo más cambios y desajuste en los datos. A esto se suma que nunca se sabe el número real de individuos infectados, solo los que se reportan, y que no hay un método efectivo de medir los recuperados de una enfermedad.

El estudio de las epidemias es muy complejo, y con el fin de solventar los problemas mencionados, se pueden realizar modificaciones a los modelos fundamentales presentados en este proyecto para conseguir ajustes mejores.

También, a medida que se descubren nuevos datos sobre las enfermedades que surgen, se mejoran los modelos. Asimismo, cada vez podemos ajustar modelos más complejos o con mayor cantidad de parámetros conforme avanza la potencia de los ordenadores para realizar simulaciones, así como podemos usar mayores cantidades de datos.

Además, dado que cada enfermedad tiene características bastante distintas a las demás, sería conveniente usar métodos de ajuste distintos para cada conjunto de datos. Pese a que, em la realización de este trabajo, con el ajuste que hemos implementado en la página web se obtengan buenos resultados en general para varios conjuntos de datos distintos y varios modelos, siempre puede mejorarse haciendo estudios más específicos de cada problema.

A pesar de las limitaciones encontradas, durante este proyecto hemos comprobado con las simulaciones realizadas cómo los ajustes de los datos son bastante buenos. En el caso de los datos del covid en España, el modelo que mejor se ajusta a estos datos es el modelo SIR. Esto se debe a la propia naturaleza de la enfermedad, pues el desarrollo de esta consiste en ser vulnerable a ser infectado, una vez infectado puedes contagiar a las personas de tu entorno y, al recuperarte, se tiene una cierta inmunidad durante un tiempo posterior. Al realizarse los ajustes por olas y durar estas menos que el tiempo de inmunidad tras superar el covid, a efectos prácticos de este modelo una vez recuperado permaneces así. Un modelo SI en este caso no ajustaría bien pues una vez infectado no permaneces infectado y contagiando a los demás indefinidamente, así como un modelo SIS tampoco, pues eres inmune al contagio durante un tiempo. Por el contrario, en el ajuste realizado con los datos del VIH, dado que una vez infectado siempre permaneces infectado y puedes producir contagios nuevos, un modelo SI se ajusta mejor que los otros, pues el modelo SIS no se adapta porque los infectados deberáin poder recuperarse. Sí sería una opcion ajustar mediante un modelo SIR, considerando como los recuperados el número de fallecidos por la enfermedad, pues ya no pueden contagiarse. Pese a esto, el modelo que mejor parece ajustarse es el modelo SI, lo que creo que es debido a la baja mortalidad del VIH en los últimos años.

Durante la realización del proyecto se han cumplido los objetivos marcados al inicio de este, y se ha profundizado en distintas áreas complementarias que facilitan la comprensión de lo presentado e ilustran la relevancia de los temas tratados. Los objetivos que finalmente se han completado son:

\begin{itemize}
\item \textbf{Descripción de modelos discretos en epidemiología}. Se han presentado y analizado diversos modelos discretos en epidemiología mostrando su comportamiento, características y particularidades de cada uno.
\item \textbf{Descripción de modelos continuos en epidemiología}. Se han presentado y analizado los modelos continuos análogos a los discretos anteriormente, explicando las diferencias entre su versión discreta y continua en cada caso.
\item \textbf{Visualización del comportamiento de los modelos}. Se han realizado representaciones gráficas de las simulaciones de los modelos con el fin de facilitar la comprensión y mostrar el comportamiento de los modelos en cada caso.
\item \textbf{Desarrollo de una página web}. Se ha desarrollado una página web donde se recoge una breve explicación de cada modelo, tanto discretos como continuos, y se pueden modificar los distintos parámetros de cada uno, obteniendo las gráficas pertinentes en cada caso.
\item \textbf{Ajuste de datos}. Se ha implementado una funcionalidad en la página web para poder ajustar ficheros de datos subidos por el usuario usando los modelos epidemiológicos ya presentados. El usuario puede elegir el modelo o el sistema determinará cuál es el modelo que mejor se ajusta a dichos datos.
\item \textbf{Ajuste de datos reales}. Usando datos de enfermedades reales, se ha realizado un ajuste con los modelos estudiados con el fin de presentar una aplicación práctica real.
\end{itemize}

\section{Vías futuras}

Como vías futuras para ampliar este trabajo tenemos muchas opciones. Partiendo de los ajustes de datos reales que hemos realizado, podemos realizar modificaciones al modelo SIR para contemplar más estados posibles de los individuos, por ejemplo añadiendo estados de cuarentena o de exposición, como se ha hecho en \cite{gutierrez2020analisis}. En \cite{RAMOS2021105937} también se propone otro modelo, aún más complejo y con más estados para el covid.

Otra posibilidad es, sin modificar los modelos con los que se ha trabajado, tratar de hacer un mejor tratamiento de los datos de los que se dispone, minimizando así el ruido de estos y mejorando los resultados.

Además, más adelante se podría realizar el ajuste con el modelo SIR de los datos del covid de forma distinta a como se hecho en este proyecto, pues no tiene en cuenta el número de recuperados (que es muy difícil de medir) y trabaja solo con los infectados. Repetir nuestro ajuste considerando solo el número de infectados podría mejorar sus resultados al disminuir el ruido que se introduce en los datos. Minimizar lo más posible el ruido es especialmente importante al trabajar en estos modelos, pues al tratarse de modelos exponenciales el error se magnifica. Un ejemplo de cómo se podría llevar a cabo este ajuste se encuentra en \cite{enrique_amaro}.

También, como ya se comentó al principio del trabajo, podemos ajustar varios modelos de acuerdo a las restricciones vigentes en cada momento o no, y comprobar cuáles son más efectivas, así como podríamos comparar usando el mismo modelo si se producen cambios significativos en la propagación de epidemias antes y después de la vacunación de la misma.








