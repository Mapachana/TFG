% Plantilla para un Trabajo Fin de Grado de la Universidad de Granada,
% adaptada para el Doble Grado en Ingeniería Informática y Matemáticas.
%
%  Autor: Mario Román.
%  Licencia: GNU GPLv2.
%
% Esta plantilla es una adaptación al castellano de la plantilla
% classicthesis de André Miede, que puede obtenerse en:
%  https://ctan.org/tex-archive/macros/latex/contrib/classicthesis?lang=en
% La plantilla original se licencia en GNU GPLv2.
%
% Esta plantilla usa símbolos de la Universidad de Granada sujetos a la normativa
% de identidad visual corporativa, que puede encontrarse en:
% http://secretariageneral.ugr.es/pages/ivc/normativa
%
% La compilación se realiza con las siguientes instrucciones:
%   pdflatex --shell-escape main.tex
%   bibtex main
%   pdflatex --shell-escape main.tex
%   pdflatex --shell-escape main.tex

% Opciones del tipo de documento
\documentclass[oneside,openright,titlepage,numbers=noenddot,openany,headinclude,footinclude=true,
cleardoublepage=empty,abstractoff,BCOR=5mm,paper=a4,fontsize=12pt,main=spanish]{scrreprt}

% Paquetes de latex que se cargan al inicio. Cubren la entrada de
% texto, gráficos, código fuente y símbolofs.
\usepackage[utf8]{inputenc}
\usepackage[T1]{fontenc}
\usepackage{fixltx2e}
\usepackage{graphicx} % Inclusión de imágenes.
\usepackage{grffile}  % Distintos formatos para imágenes.
\usepackage{longtable} % Tablas multipágina.
\usepackage{wrapfig} % Coloca texto alrededor de una figura.
\usepackage{rotating}
\usepackage[normalem]{ulem}
\usepackage{amsmath}
\usepackage{dsfont}
\usepackage{textcomp}
\usepackage{amssymb}
\usepackage{capt-of}
\usepackage[colorlinks=true]{hyperref}
\usepackage{tikz} % Diagramas conmutativos.
\usepackage{minted} % Código fuente.
\usepackage[T1]{fontenc}
\usepackage[numbers]{natbib}

% Plantilla classicthesis
\usepackage[beramono,eulerchapternumbers,linedheaders,parts,a5paper,dottedtoc,
manychapters,pdfspacing]{classicthesis}

% Geometría y espaciado de párrafos.
\setcounter{secnumdepth}{0}
\usepackage{enumitem}
\setitemize{noitemsep,topsep=0pt,parsep=0pt,partopsep=0pt}
\setlist[enumerate]{topsep=0pt,itemsep=-1ex,partopsep=1ex,parsep=1ex}
\usepackage[top=1in, bottom=1.5in, left=1in, right=1in]{geometry}
\setlength\itemsep{0em}
\setlength{\parindent}{0pt}
\usepackage{parskip}

% Profundidad de la tabla de contenidos.
\setcounter{secnumdepth}{3}

% Usa el paquete minted para mostrar trozos de código.
% Pueden seleccionarse el lenguaje apropiado y el estilo del código.
\usepackage{minted}
\usemintedstyle{colorful}
\setminted{fontsize=\small}
\setminted[haskell]{linenos=false,fontsize=\small}
\renewcommand{\theFancyVerbLine}{\sffamily\textcolor[rgb]{0.5,0.5,1.0}{\oldstylenums{\arabic{FancyVerbLine}}}}

% Path para las imágenes
\graphicspath{{figures/}}

% Archivos de configuración.
%------------------------
% Bibliotecas para matemáticas de latex
%------------------------
\usepackage{amsthm}
\usepackage{amsmath}
\usepackage{tikz}
\usepackage{tikz-cd}
\usetikzlibrary{shapes,fit}
\usepackage{bussproofs}
\EnableBpAbbreviations{}
\usepackage{mathtools}
\usepackage{scalerel}
\usepackage{stmaryrd}

%------------------------
% Estilos para los teoremas
%------------------------
\theoremstyle{plain}
\newtheorem{theorem}{Theorem}
\newtheorem{proposition}{Proposition}
\newtheorem{lemma}{Lemma}
\newtheorem{corollary}{Corollary}

\theoremstyle{definition}
\newtheorem{definition}{Definition}
\newtheorem{postulate}{Postulate}
\newtheorem*{postulate 3'}{Postulate 3'}
\newtheorem*{postulate 2'}{Projective Measurement}

\renewenvironment{proof}{{\bfseries Proof.}}{\qed}

% Change the proof style so it's in English and add \qed at the end.
\renewenvironment{proof}{{\bfseries Proof.}}{\qed}

\theoremstyle{remark}
\newtheorem{remark}{Remark}
\newtheorem{exampleth}{Example}

\begingroup\makeatletter\@for\theoremstyle:=definition,remark,plain\do{\expandafter\g@addto@macro\csname th@\theoremstyle\endcsname{\addtolength\thm@preskip\parskip}}\endgroup

%------------------------
% Macros
% ------------------------

\newcommand*{\C}{\mathds{C}}
\newcommand*{\ra}{\rangle}
\newcommand*{\la}{\langle}

% Para poner sonrisa sobre puntos suspensivos. Uso: \overplace{n}{\dotsc}
\newcommand{\overplace}[2]{%
	\overset{\substack{#1\\\smile}}{#2}%
}  % En macros.tex se almacenan las opciones y comandos para escribir matemáticas.
\input{imports/classicthesis-config} % En classicthesis-config.tex se almacenan las opciones propias de la plantilla.

% Color institucional UGR
% \definecolor{ugrColor}{HTML}{ed1c3e} % Versión clara.
\definecolor{ugrColor}{HTML}{c6474b}  % Usado en el título.
\definecolor{ugrColor2}{HTML}{c6474b} % Usado en las secciones.

% Datos de portada
\usepackage{titling} % Facilita los datos de la portada
\author{Ana Buendía Ruiz-Azuaga} 
\date{\today}
\title{Modelos Epidemiológicos}

% Portada
\include{imports/titlepage}
\usepackage{wallpaper}
\usepackage[main=spanish]{babel}


\begin{document}

\ThisULCornerWallPaper{1}{figures/ugrA4.pdf}
\maketitle
\tableofcontents

%%\ctparttext{\color{black}\begin{center}
%		Esta es una descripción de la parte de informática.
%\end{center}}

%\part{Parte de informática}

\chapter*{Resumen}


A lo largo de la historia y en la actualidad, la población en todo el mundo se ha visto afectada por enfermedades de diversa índole, ya sean nuevas o conocidas. Estas enfermedades, dependiendo de su velocidad de propagación y su gravedad, suponen un problema a la población. Si la enfermedad está ampliamente extendida hablamos de una epidemia, que puede suponer graves consecuencias de salud pública, sociales y económicas.

Con el fin de comprender el desarrollo y comportamiento de las enfermedades, así como de frenar su propagación, se han desarrollado modelos epidemiológicos, tanto discretos como continuos. En este trabajo se han estudiado modelos clásicos en epidemiología, SI, SIR y SIS, en sus versiones discretas y continuas. De cada uno de ellos se realizará un análisis teórico para estudiar su comportamiento y cómo evolucionará de acuerdo a los valores de los parámetros que caracterizan a cada uno. Se estudiarán así las condiciones necesarias para el correcto funcionamiento del modelo, garantizando la positividad de las soluciones, y el desarrollo esperado estudiando puntos de equilibrio y la estabilidad de estos, así como tratar de prever si se producirá una epidemia de acuerdo a los valores de los parámetros dados usando el número básico reproductivo.

Además, se implementará software que permite obtener representaciones gráficas del comportamiento de cada modelo de acuerdo a parámetros introducidos por el usuario, facilitando así la visualización e interpretación de estos modelos.

Asimismo se incluirá una funcionalidad para realizar ajustes de datos de acuerdo a distintos modelos, siendo el sistema capaz de determinar cuál es el modelo que mejor se ajusta a los datos seleccionados. Para cada conjunto de datos y modelo escogidos se muestra la representación gráfica del ajuste, los parámetros estimados y el error obtenido mediante el ajuste realizado.

Finalmente, usando el software desarrollado se ajustarán datos reales de diversas enfermedades de interés en la actualidad, comprobando qué modelos son mejores para describir el comportamiento de cada enfermedad.

\textbf{Palabras clave: } Modelo discreto, modelo continuo, modelo SI, modelo SIR, modelo SIS, propagación, epidemia, simulación.



\chapter*{Summary}

Throughout history and nowadays, the population around the world has been affected by diseases of various kinds, those being new or already known at the moment. These diseases, depending on their propagation speed and severity, can be a problem to the population. If a disease has widely spread, we talk about an epidemic. An epidemic can provoke a severe impact on public health, society and the economy.

In an effort to understand the development and behavior of the disease, as well as slow down their spreading, both discrete and continuous models started to be used. A theoretical analysis of each of them will be done with the purpose of studying the behavior of the disease and how they will evolve according to the values of the parameters which characterize each one of them. Thus, necessary conditions for the correct performance of the model and the expected development will be studied, as well as trying to predict if an epidemic will take place according to the values of the parameters.

In addition to this, software to obtain a graphic representation of the behavior of each model according to the parameters values established by the user will be implemented. This way, visualization and interpretation of the models are more accessible to the user.

Furthermore, a functionality to fit data according to different models, with the system being able to determine which one of the available models is best for the selected data, will be implemented. For each dataset to fit and each model, the software will show a graphic representation of the fit, the value of the fitted parameters and the estimated fitting error.

Finally, using the software developed before, some real data from different diseases with high relevance nowadays will be fitted, determining which models fit best for describing their behavior.

\section*{Introduction}

Contagious diseases such as the flu or tuberculosis can cause significant problems to a city or region. Recent epidemics such as Covid-19 and aids have had a huge impact on society nowadays. The behavior of contagious diseases is especially important in less developed countries and regions due to high mortality and lack of medical resources. This causes a shorter life expectancy and a significant impact on the economy.

Throughout history, epidemics have affected civilization development. There are books that refer to epidemics as plagues, they contain information about how the Antonine Plague contributed to the fall of the Roman Empire or similarly did the Han Dinasty. The bubonic plague also caused severe problems in Europe and China during the XIV century.

When discussing contagious diseases, many questions arise: ''How severe will the epidemic be?'', ''How many people will get sick?'', ''How long will the epidemic last?''

Experiments to obtain information about epidemics are really difficult to make due to the irreproducibility of the context, so only naturally caused epidemics are studied. This implies that data is incomplete and not accurate.

Mathematical models in epidemiology help understand the underlying mechanisms of disease propagation as well as suggest different techniques to improve disease control.

\section*{Objectives}

In this work, different mathematical models have been used to simulate epidemics and their behavior. The simulations are included on a website that has been developed as a part of this project. We have tried to make a theoretical exhibition and analysis of mathematical models as well as show its importance in the real world by implementing simulations and data fits with real data from diseases.

The objectives that have been accomplished are:

\begin{itemize}
\item Discrete mathematical models in epidemiology. Discrete mathematical models have been studied, finding their fixed points, their stability and overall expected behavior using the reproductive number. The particularities, parameters and characteristics of every model have been explained.
\item Continuous mathematical models in epidemiology. Continuous versions of the discrete mathematical models studied before have been studied, finding their fixed points, their stability and overall expected behavior using the reproductive number. The particularities, parameters and characteristics of every model have been explained, as well as their differences with their discrete version.
\item Visualization of model's behavior. Graphic representations of the simulations of every model have been made in order to help understand and show the behavior of models.
\item Website development. A website has been implemented where you can find a brief explanation of each model, discrete or continuous, and parameters values can be modified, showing graphics of the behavior according to the given values of the parameters.
\item Data fit. There is a functionality on the website for the user to upload their data files and fit a model. The model to fit can be chosen by the user or the system can determine which model fits the data better.
\item Real data fit. Using real data of current epidemics we have fitted the data with the studied models to present a real use of this project.
\end{itemize}

\section*{Description}

This project contains a theoretical explanation of discrete and continuous mathematical models in epidemiology, analyzing the SI, SIR and SIS models, their behavior, fixed points and their stability, necessary and sufficient conditions to guarantee positivity of solutions and particular characteristics of each of them. 

Furthermore, a website has been implemented so the users can have easy access to a brief explanation of each model and modify the value of the parameters of each model, refreshing graphics that show the behavior of the model according to the values of the given parameters. This way, the user can graphically and intuitively see the behavior of the model and understand it.

On the website, the user also can upload data files and obtain a fit for them using the studied models. The user can manually choose which model use for the data fit or let the system determine which model fits them better. The data fit provides an approximation for the values of the parameters for the selected model and the error obtained when using those values to describe the data selected. This way, data fits can be done by simply formatting the data file to the specified format.

\section*{Structure}

In this document we can see several differentiated parts:

\begin{itemize}
\item Bases and necessary tools.
\item Discrete mathematical models in epidemiology.
\item Continuous mathematical models in epidemiology.
\item Software design and analysis.
\item Real data fit.
\item Discussion and future work.
\end{itemize}

\subsection*{Bases and necessary tools}

In this chapter, some fundamental concepts and propositions that are necessary in order to understand the next chapters are explained.  These tools will be used several times during the development of the project.

\subsection*{Discrete mathematical models in epidemiology}

Discrete mathematical models in epidemiology are defined and studied, finding their fixed points and their stability. Their characteristics and how the values of their parameters impact their behavior are studied and simulated. In this chapter, the reproductive number is also defined and used to try to predict if an epidemic is going to happen or not.

\subsection*{Continuous mathematical models in epidemiology}

Continuous versions of the previous discrete mathematical models in epidemiology are defined and studied, finding their fixed points and their stability. Their characteristics and how the values of their parameters impact their behavior are studied and simulated. The difference between this model and its discrete versions will be explained in each model.

\subsection*{Software design and analysis}

In this chapter, the documentation of the website development will be presented, including resource management and time planification of the project. The requisites of the software are detailed and the use cases. A conceptual diagram, used tools and frameworks, prototypes and the user manual can be found here too.

\subsection*{Real data fit}

Using the already developed software that is detailed in the previous chapter, a data fit using real data from Covid-19 and vih diseases is done, determining in each case which model fits each disease. By doing this, we calculate an approximation of the values of the parameters for each model and the error obtained by using that value of the parameters to describe the real data. A graphic representation of the data is also provided in order to help the visualization of the fit and its accuracy.

\subsection*{Discussion and future works}

Lastly, we discuss different topics about the project, such as the assumptions that are used by the models and if they are satisfied in the real data collected or the problems that we can face when trying to collect data from a disease. The accomplished objectives, improvements to this project that can be done or additional works to deepen on this field are also talked about in this chapter.

\textbf{Keywords: } Discrete model, continuous model, SI model, SIR model, SIS model, propagation, epidemic, simulation.



% Part I
\chapter{Herramientas básicas}

\section{Tipos de modelos}

Los modelos discretos (por ejemplo SI, SIR y SIS) usan los estados Susceptible, Infectado y Recuperado. Los nombres suelen hacer referencia al flujo que se sigue para pasar entre los estados. Así, por ejemplo un modelo SI pasa de susceptible a infectado, uno SIR de susceptible a infectado y recuperado y SIS alterna entre susceptible e infectado.

En estos modelos se hacen dos suposiciones:
\begin{enumerate}
\item La población se mezcla de manera homogénea, es decir, todos los individuos tienen la misma probabilidad de contraer la enfermedad.
\item El total de la población es constante.
\end{enumerate}

\subsection{Modelo SI}
Es el modelo más simple de todos, los individuos nacen siendo susceptibles a una enfermedad, una vez infectados no hay tratamiento y permanecen infectados el resto de su vida.
Un ejemplo de una enfermedad que pueda modelarse usando SI es el herpes.

Las siguientes ecuaciones describen el modelo SI:

\begin{equation}
\label{eqn: SI}
\begin{aligned}
S_{n+1}=S_n\left( 1-\frac{\alpha\Delta t}{N}I_n\right) \\
I_{n+1}=I_n\left( 1+\frac{\alpha\Delta t}{N}S_n\right)
\end{aligned}
\end{equation}

donde $S_n$ indica el número de individuos susceptibles en el instante $t_n$, así como $I_n$ hace referencia al número de individuos infectados en ese instante. $\Delta t$ es el tiempo transcurrido entre dos instantes $t_{n+1}-t_n$ y N es el tamaño total de la población, con condiciones iniciales $S_0>0$, $I_0>0$ y $S_0+I_0=N$.

En estas ecuaciones $\alpha$ es la tasa de contacto, esto es, el número medio de individuos con los que un infectado tiene suficiente contacto para contagiarlo en un intervalo de tiempo. Por tanto, $S_n$ representa el número de individuos susceptibles en el tiempo $n\Delta t$.

Ahora, imponemos las suposiciones descritas anteriormente para estos modelos. En primer lugar, suponemos que la población se mezcla de manera homogénea de ahora en adelante, y para la segunda: La población total se mantiene constante, que es trivial que se cumpla siempre, ya que sumando el sistema de ecuaciones el resultado es $N$ y asumimos que las soluciones son siempre positivas pues las soluciones negativas no tienen sentido.
Además, no tiene sentido hablar de un número negativo de individuos, ya sean infectados, recuperados o susceptibles de contraer la enfermedad, luego para imponer que las ecuaciones tengan soluciones positivas una condición necesaria y suficiente para $S_n$ es que $\alpha\Delta t \leq 1$.

\begin{proof}
Supongamos que $I_n, S_n > 0$. Por la segunda ecuación del modelo (\ref{eqn: SI}) es claro que $I_{n+1}>0$, pues $S_n>0$ y $1+\frac{\alpha\Delta t}{N}>0$.

Para la primera ecuación tenemos que 
$$S_{n+1}>0 \Leftrightarrow 1-\frac{\alpha\Delta t}{N}I_n >0$$

ya que $I_n$ es positivo por hipótesis. Esto equivale a:
$$N>\alpha\Delta t(N-I_n) \Leftrightarrow \alpha\Delta t I_n > (\alpha\Delta t -1) N$$

Como $\alpha\Delta t I_n > 0$, tenemos entonces que la desigualdad se da si y solo si:
$$\alpha\Delta t -1 <0 \Leftrightarrow \alpha\Delta t \leq 1$$
\end{proof}


Buscamos ahora ver cuál es el comportamiento del sistema, calculando los puntos de equilibrio, para lo que resolvemos:

$$
\begin{cases}
S^*=S^*\left( 1-\frac{\alpha\Delta t}{N}I^*\right) \\
I^*=I^*\left( 1+\frac{\alpha\Delta t}{N}S^*\right) \\
S^*+I^*=N
\end{cases}
$$

Los únicos puntos de equilibrio posibles son: $S^*=0, I^*=N$ y $S^*=N, I^*=0$, y como sabemos que tenemos condiciones iniciales positivas y es claro que $S_n$ es monótonamente decreciente e $I_n$ es monótonamente creciente, ya que $S_{n+1}$ es $S_n$ multiplicado por un valor menor que $1$, mientras que $I_{n+1}$ corresponde a $I_n$ multiplicado por un valor mayor que $1$, así $S_{n+1}<S_n$ y $I_{n+1}>I_n$ para cualquier $n\in\mathbb{N}$, entonces debe converger a $S^*=0, I^*=N$, pues son sucesiones monótonas acotadas.

Expresando $\alpha$ como una tasa podemos obtener las ecuaciones diferenciales análogas de la siguiente manera:

$$\frac{S_{n+1} - S_n}{\Delta t} \approx S'(t),$$

luego su análoga continua es:

\begin{equation}
\begin{aligned}
S'(t) = -\frac{\alpha}{N}SI \\
I'(t) = \frac{\alpha}{N}SI
\end{aligned}
\end{equation}

con condiciones iniciales $S(0)+I(0)=N$.

De manera análoga al caso discreto, es decir, suponiendo las funciones constantes y resolviendo el sistema de ecuaciones, podemos comprobar que este sistema converge a $S^*=0, I^*=N$ y, por tanto, tiene el mismo comportamiento que el caso discreto.
\textcolor{red}{RESOLVER, VIENE EN EL ARTICULO}

\begin{figure}
\begin{center}
\caption{Gráfica del modelo SI, en una población total de $100$ individuos, con valores iniciales $S_0=99, I_0 = 1, \alpha = 0.1, T_0 = 0, T = 100$.}
\includegraphics[scale=1]{graficaSI}
\end{center}
\end{figure}

\subsection{Modelo SIR}
Comienza como el SI, pero tras infectarse los individuos pasan a un estado Recuperado, en el cual no pueden infectarse ni infectar a otros.
Por ejemplo, la varicela. 

El modelo es el siguiente:

\begin{equation}
\label{eqn: SIR_modelo}
\begin{aligned}
S_{n+1} = & S_n \left(1-\frac{\alpha\Delta t}{N} I_n \right) \\
I_{n+1} = & I_n \left( 1-\gamma \Delta t + \frac{\alpha\Delta t}{N} S_n \right) \\
R_{n+1} = & R_n + \gamma \Delta t I_n
\end{aligned}
\end{equation}

con condiciones iniciales $S_0>0$, $I_0>0$, $R_0\geq 0$, satisfaciendo $S_0+I_0+R_0=N$.

En estas ecuaciones, de nuevo tenemos que $\alpha$ es la tasa de contacto, esto es, el número medio de individuos con los que un infectado tiene suficiente contacto para contagiarlo en un intervalo de tiempo y $\gamma$ es la probabilidad de que un infectado pase a recuperado/retirado/aislado/fallecido en un intervalo de tiempo, con $\alpha >0$ y $\gamma >0$.

Se supone que la población permanece constante, $S_n+I_n+R_n=N$.

Además, tenemos que las soluciones a este sistema discreto son positivas para cualquier valor de las condiciones iniciales si, y solo si:

$$\max{\big\{\gamma\Delta t, \alpha\Delta t\big\} } \leq 1$$

\textcolor{red}{Aqui falta la demostracion de esto}

o equivalentemente:

$$\min{\bigg\{ \frac{1}{\gamma}, \frac{1}{\alpha} \bigg\} } \geq \Delta t$$

Por tanto, el intervalo de tiempo debe ser menor que el tiempo medio requerido para un contacto exitoso y menor que el período medio infeccioso.
% Con contacto exitoso asumo que se refiere al tiempo necesario para infectar a un individuo. Sí es esto confirmado por la profe

El comportamiento global del sistema es fácil de ver. Definimos como $\mathcal{R}=S_0 \alpha/(N\gamma )$ la tasa reproductiva. El valor de $\mathcal{R}$ determina el comportamiento global del modelo. \textcolor{red}{(a esto después lo llamo tasa de transmisión media en el artículo \cite{demongeotSIEpidemicModel})}. 

Notemos que $S_n$ es estrictamente decreciente y $R_n$ es estrictamente creciente. Estudiémoslas:

Sea $S_\infty=\lim_{n\rightarrow\infty} S_n\geq 0$, cuyo límite existe pues es una sucesión estrictamente decreciente y acotada inferiormente por $0$, que depende de las condiciones iniciales. Si $S_0\leq \frac{\gamma N}{\alpha}$, o, equivalentemente, $\mathcal{R}\leq 1$ entonces $I_1\leq I_0$ y, como $S_n$ es estrictamente decreciente, tenemos que $I_{n+1}\leq I_n$, es decir, no hay epidemia. En otro caso, tenemos $S_0> \frac{\gamma N}{\alpha}$, entonces $I_1>I_0$. Debe ocurrir que $S_\infty <\frac{N\gamma}{\alpha}$, pues si no fuera así, tendríamos que $I_n$ crece hacia un equilibrio, $I_\infty$, lo que implica que $R_n$ se aproxima a infinito cuando $n\rightarrow\infty$, lo cual no es posible. Así, el número de infectados finalmente comienza a decrecer y se aproxima a $0$. Además, sabemos por el Lema 1 de \cite{allenDiscretetimeSISIR1994} que $S_\infty>0$.

El modelo continuo se comporta de la misma forma que el modelo discreto, este sería:

\begin{equation}
\label{eqn: modelo_SIR_continuo}
\begin{cases}
S'(t) = -\dfrac{\alpha}{N}SI \\
I'(t) = I\left(d\frac{\alpha}{N}S-\gamma \right) \\
R'(t) = R+\gamma I
\end{cases}
\end{equation}

donde $S(0)+I(0)+R(0)=N$. La tasa reproductiva en este caso es $\mathcal{R}=S(0)\alpha /(N\gamma )$, y si $\mathcal{R}\leq 1$  no hay epidemia, pero en cambio, si es mayor, hay epidemia.

\begin{figure}
\begin{center}
\caption{Gráfica del modelo SIR, en una población total de $100$ individuos, con valores iniciales $S_0=99, I_0 = 1, R_0 = 0, \alpha = 0.1, \gamma = 0.01 T_0 = 0, T = 300$.}
\includegraphics[scale=1]{graficaSIR}
\end{center}
\end{figure}


\subsection{Modelo SIS}
Es similar al SI, pero tras infectarse los individuos vuelven a ser susceptibles.
Por ejemplo, los resfriados pueden modelarse usando SIS.

El modelo es una perturbación del modelo SI visto antes, es el siguiente:

\begin{equation}
\label{eqn: modelo_SIS}
\begin{aligned}
S_{n+1} = S_n \left(1-\frac{\alpha\Delta t}{N} I_n \right) + \gamma \Delta t I_n \\
I_{n+1} = I_n \left( 1-\gamma \Delta t + \frac{\alpha\Delta t}{N} S_n \right)
\end{aligned}
\end{equation}

con condiciones iniciales positivas $S_0>0$, $I_0>0$ cumpliendo $S_0+I_0=N$. Por lo tanto, el tamaño de la población es constante.

En estas ecuaciones, $\alpha$ de nuevo representa la tasa de contacto, esto es, el número medio de individuos con los que un infectado tiene suficiente contacto para contagiarlo en un intervalo de tiempo y $\gamma$ es la probabilidad de que un infectado pase a recuperado/retirado/aislado/fallecido en un intervalo de tiempo, donde se cumple $\alpha >0$ y $\gamma >0$.

Además las soluciones siempre son positivas si, y solo si:

$$\gamma \Delta t \leq 1 $$ y $$\alpha\Delta t< \left( 1+\sqrt{\gamma \Delta t} \right)^2$$

Esto se puede consultar en el Lema 2 del Apéndice de \cite{allenDiscretetimeSISIR1994}.

\textcolor{red}{No he comprobado las cuentas de esas condiciones}

La tasa reproductiva de este modelo se define como $\mathcal{R}=\alpha /\gamma$ \textcolor{red}{No entiendo como saca esa tasa reproductiva, he repetido las cuentas y las de antes estaban mal. Si no es por definición como dijimos el otro día no se dónde sale}

Si $\mathcal{R}\leq 1$ entonces se tiene que $I_{n+1} < I_n$, ya que $0<S_n<N$ y las soluciones son positivas. En este caso es fácil ver que el límite, al ser una sucesión monótona decreciente y acotada inferiormente, es $(S^*,I^*)=(N,0)$. Supongamos que $S^*<N$, entonces existen $n_1, \epsilon$ tales que para todo $n \geq n_1$:
$$S_n<S^*+\epsilon < N$$
y usando las ecuaciones (\ref{eqn: modelo_SIS})
$$I_{n+1} \leq I_n \left( 1-\gamma \Delta t + \frac{\alpha\Delta t}{N} S_n \right) = \rho I_n$$

Como $\rho < 1$ tenemos que $I^*=0$, lo que contradice que $S^*<N$.

Si $\mathcal{R}>1$ realizando la sustitución $S_n=N-I_n$ y el cambio

$$x_n=\frac{\alpha \Delta t I_n}{N(1+\alpha \Delta t - \gamma \Delta t)}$$

tenemos:

\begin{equation}
\begin{aligned}
x_n=\frac{\alpha \Delta t I_n}{N(1+\alpha \Delta t - \gamma \Delta t)} \Leftrightarrow \\
\alpha\Delta t I_n = x_nN(1+\alpha\Delta t-\gamma\Delta t) \Leftrightarrow \\
I_n = \frac{x_nN(1+\alpha\Delta t - \gamma\Delta t}{\alpha\Delta t}
\end{aligned}
\end{equation}

entonces sustituyendo:

\begin{equation}
\begin{aligned}
\frac{x_{n+1}N(1-\alpha\Delta t-\gamma\Delta t)}{\alpha \Delta t} = \frac{x_nN(1+\alpha\Delta t-\gamma \Delta t)}{\alpha\Delta t}\left( 1-\gamma\Delta t+\frac{\alpha\Delta t}{N}\left(N-\frac{x_nN(1+\alpha\Delta t-\gamma\Delta t)}{\alpha\Delta t}\right) \right)
\end{aligned}
\end{equation}

Despejando de esta expresión:

\begin{equation}
\begin{aligned}
x_{n+1} & = x_n\left( 1-\gamma\Delta t+\frac{\alpha\Delta t}{N}N-\frac{\alpha\Delta t}{N}\frac{x_nN(1+\alpha\Delta t -\gamma \Delta t)}{\alpha\Delta t} \right) \\
& = x_n(1-\gamma\Delta t + \alpha\Delta t -x_n(1+\alpha\Delta t -\gamma\Delta t)) \\
& = x_n((1-\gamma\Delta t+\alpha\Delta t)(1-x_n))
\end{aligned}
\end{equation}

luego obtenemos la ecuación logística

$$x_{n+1} = (1+\alpha \Delta t - \gamma \Delta t)x_n(1-x_n)$$

Entonces la restricción necesaria para garantizar soluciones positivas no es suficiente para asegurar la convergencia, en este caso a dicha restricción hay que añadir la condición $\alpha \Delta t \leq 2+\gamma \Delta t$ \textcolor{red}{Esta condición supongo que viene de alguna condición conocida de la ecuación logística que no recuerdo, ups. A partir de aquí me pierdo un poco, está en el último párrafo de la página 10 del artículo de Allen}


 







\section{Cosas del articulo de rsos que no sé que nombre ponerles}

\subsection{Introducción}

Estimar la tasa de transmisión media es uno de los aspectos más cruciales en epidemiología. Esta tasa condiciona la fase de la epidemia e incluso si va a extinguirse. Es combinación de tres factores:

\begin{enumerate}
\item Coeficiente de virulencia: Relacionado con el agente infeccioso.
\item Coeficiente de susceptibilidad: Relacionado con el anfitrión.
\item Número de contactos por unidad de tiempo entre individuos.
\end{enumerate}

Los dos primeros factores se tienen en cuenta a la vez en la probabilidad de transmisión.

Todos los factores pueden cambiar con el tiempo, el primero debido a mutaciones del virus y los dos últimos por medidas de contención. Por tanto, observar el decrecimiento de la transmisión media en una enfermedad es una buena forma de comprobar la efectividad de las medidas de contención.

Consideramos un modelo SI modificado con el objetivo de compararlo con los datos obtenidos en la pandemia de la COVID-19 hasta el momento y así tratar de predecir su comportamiento en el futuro.

El modelo SI continuo es el siguiente:

\begin{equation}
\label{eqn: SI_cont}
\begin{aligned}
S'(t) = -\tau (t)S(t)I(t) \\
I'(t) = \tau (t)S(t)I(t) -vI(t)
\end{aligned}
\end{equation}

donde $S(t)$ es el número de individuos susceptibles , $I(t)$ el número de individuos infectados en el tiempo $t$ y $\tau (t)$ la tasa de transmisión, que combina el número de contactos por unidad de tiempo y la probabilidad de transmisión. Además, notemos que $v$ es constante, donde $1/v$ es la duración media del período de infección, y $vI(t)$ el flujo de individuos recuperados o fallecidos. %vI(t) es el flujo de recuperados o muertos porque ha mezclado el modelo SI y SIR; vI(t) serian los recuperados, pero se ahorra la ecuacion de la R

$$S(t_0)=S_0>0, \: I(t_0)=I_0>0$$

Ahora, consideramos que al final del período infeccioso nos han informado de una fracción del total de casos, en este caso $f\in (0,1]$. Sea $C_R(t)$ el número total (acumulado) de casos reportados. Entonces:

\begin{equation}
\label{eqn: acumulada}
C_R(t) = {C_R}_0 + vfC_I(t) \; \forall t \geq t_0
\end{equation}

\textcolor{red}{¿Por qué $vfC_I(t)$ va multiplicado por $v$?}

donde

$$C_I(t) = \int_{t_0}^t I(w) dw $$

Asumimos conocidos $S_0 > 0$, $1/v>0$, $f\in (0,1]$. Por tanto, queremos averiguar $I_0$, $\tau (t)$.

\subsection{Aproximando $I_0$ y $\tau (t_0)$}
Ahora, procedemos a intentar aproximar $I_0$ y $\tau (t_0)$:

Al comienzo de la pandemia podemos asumir que $S(t)$ y $\tau (t)$ son constantes e iguales a $S_0$ y $\tau_0 = \tau (t_0)$ respectivamente. Así, sustituyendo estos valores en la ecuación (\ref{eqn: SI_cont}) obtenemos:

$$I'(t) = (\tau_0 S_0 -v) I(t)$$.

Resolviendo la ecuación diferencial llegamos a:

$$I(t) = I_0\exp{((\tau_0 S_0-v)(t-t_0))}$$.

Sustituyendo en (\ref{eqn: acumulada}):

$$C_R(t) = {C_R}_0 + vfI_0\frac{\mathrm{e}^{(\tau_0 S_0 -v)(t-t_0)} -1}{\tau_0 S_0-v}$$

Así, hemos obtenido un primer modelo para los casos acumulados al principio de la pandemia.

Reescribimos la ecuación como:

\begin{equation}
\label{eqn: acumulada_modelo}
C_R(t) = \chi_1 \mathrm{e}^{\chi_2 t} -\chi_3
\end{equation}

Estimamos $\chi_3$ usando los datos de la epidemia obtenidos, y el mejor ajuste para los datos es $\chi_3=0$.

Ahora, usando (\ref{eqn: acumulada}) y (\ref{eqn: acumulada_modelo}) tenemos:

\begin{equation}
I_0=\frac{\chi_1\chi_2\mathrm{e}^{\chi_2 t_0}}{vf}
\end{equation}

Y, como de reescribir sabemos que $\chi_2 = \tau_0 S_0-v$, entonces

\begin{equation}
\tau_0 = \frac{\chi_2+v}{S_0}
\end{equation}

Si suponemos que $\tau (t) = \tau_0$ constante, tenemos que el modelo queda:

\begin{equation}
\begin{aligned}
S'(t) = -\tau_0S(t)I(t) \\
I'(t) = \tau_0S(t)I(t) -vI(t)
\end{aligned}
\end{equation}

Usando la ecuación de $S(t)$ y resolviéndola obtenemos:

$$S(t) = S_0\exp{\left( -\tau_0 \int_{t_0}^t I(w) dw \right)} = S_0\exp{(-\tau_0 C_I(t))}$$

Ahora, sustituyendo esta expresión en la ecuación de $I(t)$ del modelo y usando $C_I'(t)=I(t)$:

$$I'(t) = S_0\exp{\left( -\tau_0 C_I(t)\right) }\tau_0 C_I'(t)-vI(t)$$

Finalmente, integrando entre $t_0$ y $t$ tenemos que:

$$I(t)=C_I'(t)=I_0+S_0(1-\exp{(-\tau_0 C_I(t)}))-vC_I(t)$$
 
Observamos entonces que el número total de infectados es monótono creciente, ya que $I(t)>0$ siempre por positividad de las soluciones y $C_I'(t)=I(t)>0$. Cabe destacar que esto no implica que el número de infectados sea monótono creciente.

\begin{theorem}
Sea $t>t_0$ fijo. El número de infectados acumulados es estrictamente creciente respecto a las siguiente cantidades:
\begin{itemize}
\item $I_0>0$ Número inicial de infectados
\item $S_0>0$ Número inicial de individuos susceptibles.
\item $\tau>0$ Tasa de transmisión
\item $1/v$ Tiempo medio de la infección.
\end{itemize}
\end{theorem}

\subsection{Fórmula teórica para $\tau (t)$}

Usando la ecuación del modelo inicial (\ref{eqn: SI_cont}) obtenemos:
% La ecuacion de la S$

$$S(t) = S_0 \exp{\left( - \int_{t_0}^t \tau(w) I(w) dw \right) } $$ 

Ahora, sustituyendo en la ecuación (\ref{eqn: SI_cont}):
% La ecuacion de la I

$$I'(t) = S_0 \exp{\left( - \int_{t_0}^t \tau(w) I(w) dw \right) } \tau (t) I(t) -vI(t) $$

Integramos en ambos lados entre $t_0$ y $t$, luego:

$$ C_I'(t) = I_0 + S_0 \left( 1-\exp{\left(- \int_{t_0}^t \tau (w) I(w)dw \right)}\right) -vC_I(t)$$

Equivalentemente, por (\ref{eqn: acumulada}):

$$C_R'(t) = vf\left( I_0 + S_0 \left( 1-\exp{\left(- \frac{1}{vf}\int_{t_0}^t \tau (w ) I(w)dw \right)}\right)\right) +v{C_R}_0 -vC_R(t)$$

\textcolor{red}{Esta cuenta no termina de salirme, pero tiene más o menos sentido}

Así, obtenemos el Teorema 3.1 de \cite{demongeotSIEpidemicModel}, que nos da la relación directa buscada.

\textcolor{blue}{He estado leyendo el resto del artículo, en el que habla de obtener una expresión explícita de $\tau (t)$ e $I_0$, pero para eso usan datos de China en específico y ajustes por ordenador, así que no tengo muy claro si merece la pena incluir esa parte ya que no tenemos los datos y por tanto sería fiarse un poco de lo que han hecho ellos}










% Part II
%\input{chapters/chapter5}

% ----------------------- %
% BIBLIOGRAFÍA
% ----------------------- %

% Estilo de cita.
\bibliographystyle{unsrtnat}

%[citestyle=numeric]

% Añadimos la bibliografía al índice
\phantomsection
\addcontentsline{toc}{chapter}{Bibliografía}

\bibliography{bib/library}

\end{document}