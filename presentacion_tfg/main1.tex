\documentclass[xcolor=dvipsnames]{beamer} % dvipsnames gives more built-in colors

%\usetheme{Dresden}
%\usecolortheme{beaver}
\setbeamercolor{itemize item}{fg=darkred!80!black}


%\usecolortheme{default}

%\usefonttheme{structurebold}

%\colorlet{beamer@blendedblue}{cyan!100!black}


%\setbeamertemplate{blocks}[rounded][shadow=true] 
%\setbeamertemplate{navigation symbols}{}  


%\setbeamertemplate{navigation symbols}{}  

\usetheme{Madrid}
%\useoutertheme{miniframes} % Alternatively: miniframes, infolines, split
%\useinnertheme{circles}

%\definecolor{miazul}{rgb}{0.30,0.129,0.176} % UBC Blue (primary)
%\definecolor{miazul}{rgb}{0, 0.249, 0.602} % UBC Blue (primary)

%\usecolortheme[named=miazul]{structure}
%\usecolortheme[named=Mahogany]{structure} % Sample dvipsnames color


\usepackage{times}

\usepackage{amsmath,amssymb}
\usepackage[latin1]{inputenc}

\usepackage{array}
\usepackage{amsmath}
\usepackage{amsthm}
\usepackage{amssymb}
\usepackage{graphicx}
\usepackage{hyperref}
\usepackage{longtable}
\usepackage{pdfpages}
\usepackage{eurosym} 
\usepackage{xcolor}

\usepackage[absolute,overlay]{textpos}
\usepackage{everypage}

\usepackage{pgf}
\usepackage{tikz}


\usepackage{setspace}  
\usepackage{colortbl}
\usepackage{graphicx}

\DeclareGraphicsExtensions{.jpg,.pdf,.mps,.png}




\title[Title Without Rambling]{My Rambling Presentation Title}
\date{\today}
\author[R.A.]{Rambling Academic}
\institute[RamblingAcademic.com]{RamblingAcademic.com\\Nuts and Bolts of Research. Plus Some Rambling}

\begin{document}

\begin{frame}
	\titlepage
\end{frame}

\section{Rambling Section}

\subsection{First Subsection}
	
\begin{frame}
	Here is some rambling text
	\begin{enumerate}
		\item List item 1
		\item List item 2
	\end{enumerate}
\end{frame}

\begin{frame}{First order linear difference equation}

	\transwipe
	
	\begin{textblock*}{12cm}(0.5cm,1cm)
	
	\begin{block}{General expression (constant coefficients)}
	For $\alert{\alpha}, \textcolor{blue}{\beta} \in \mathbb{C}$, $\quad n\geqslant 0$,
	\begin{equation*}
	x_{n+1} = \alert{\alpha} \, x_n + \textcolor{blue}{\beta}.
	\end{equation*}
	Eventually, $x_0$ is given as the initial term.
	\end{block}
	
	\end{textblock*}
	
	\pause
	
	\vspace{3cm}
	
	\alert{Unknown}: The sequence $\{x_n\}_{n\geqslant0}$ of complex numbers.
	
	\vspace{1cm}
	
	\alert{Objective}: To find the unknown $\{x_n\}_{n\geqslant0}$ (if it exists).
	
	
	\vspace{1cm}
	
	\alert{Particular case}. If $\alert{\alpha} =0$, the solution is the constant sequence $\{x_n\}_{n\geqslant0} = \{\beta\}_{n\geqslant0}.$
	
	
	\end{frame}
	
	%%%%%%%%%%%%%%%%%%%%%%%%%%%%%%%%%%%%%%%%%%%%%%%%%%%%%%%%%5
	
	\begin{frame}{First order linear difference equation}
	
	\transwipe
	
	\begin{textblock*}{12cm}(0.5cm,1cm)
	
	\begin{block}{General expression (constant coefficients)}
	For $\alert{\alpha}, \textcolor{blue}{\beta} \in \mathbb{C}$, $\quad n\geqslant 0$,
	\begin{equation*}
	x_{n+1} = \alert{\alpha} \, x_n + \textcolor{blue}{\beta}.
	\end{equation*}
	Eventually, $x_0$ is given as the initial term.
	\end{block}
	
	\end{textblock*}
	
	\vspace{3cm}
	
	Particular case:  \textcolor{red}{Arithmetic progression}.  For $\alert{\alpha} = 1$,
	$$
	x_{n+1} = x_n + \textcolor{blue}{\beta}, \quad n\geqslant 0.
	$$
	
	\vspace{1cm}
	
	Solution: \quad $x_{n} = x_0 + n\,\textcolor{blue}{\beta}, \quad n\geqslant 0.$
	 
	\end{frame}
	
	%%%%%%%%%%%%%%%%%%%%%%%%%%%%%%%%%%%%%%%%%%%%%%%%%%%%%%%%%5
	
	\begin{frame}{First order linear difference equation}
	
	\transwipe
	
	\begin{textblock*}{12cm}(0.5cm,1cm)
	
	\begin{block}{General expression (constant coefficients)}
	For $\alert{\alpha}, \textcolor{blue}{\beta} \in \mathbb{C}$, $\quad n\geqslant 0$,
	\begin{equation*}
	x_{n+1} = \alert{\alpha} \, x_n + \textcolor{blue}{\beta}.
	\end{equation*}
	Eventually, $x_0$ is given as the initial term.
	\end{block}
	
	\end{textblock*}
	
	\vspace{3cm}
	
	
	Particular case: \textcolor{red}{Geometric progression}.  For $\textcolor{blue}{\beta} = 0$, 
	$$
	x_{n+1} = \alert{\alpha}\, x_n, \quad n\geqslant 0.
	$$
	
	\vspace{1cm}
	
	Solution: \quad $x_{n} = x_0 \, \alert{\alpha}^n, \quad n\geqslant 0.$
	
	
	\end{frame}
	
	%%%%%%%%%%%%%%%%%%%%%%%%%%%%%%%%%%%%%%%%%%%%%%%%%%%%%%%%%%%%%%%%%%%%%%%%%%%%%%%%
	
	\begin{frame}{Solving the first order linear difference equation}
	
	\transwipe
	
	\begin{textblock*}{12cm}(0.5cm,1cm)
	
	\begin{block}{Proposition 1} Let $\alert{\alpha}, \textcolor{blue}{\beta} \in \mathbb{C}$, and $\alert{\alpha} \neq 1$. Then, the difference equation
	\begin{equation}\label{eq-lin}
	x_{n+1} = \alert{\alpha} \,x_n + \textcolor{blue}{\beta}, \quad n\geqslant 0,
	\end{equation}
	has a unique constant solution in the form $\displaystyle{x_{*} = \frac{\textcolor{blue}{\beta}}{1-\alert{\alpha}}}$
	\end{block}
	
	\end{textblock*}
	
	\vspace{4cm}
	
	The value $x_*$ is usually called \textcolor{red}{equilibrium}.
	
	\vspace{1cm}
	
	\textcolor{blue}{Proof.} A constant solution satisfies $x_n = x_*$, $n\geqslant0$. Substituting in \eqref{eq-lin}, we get
	$$
	x_{*} = \alert{\alpha} \,x_* + \textcolor{blue}{\beta}.
	$$
	Grouping terms, we get the announced expression.
	\end{frame}
	
	
	%%%%%%%%%%%%%%%%%%%%%%%%%%%%%%%%%%%%%%%%%%%%%%%%%%%%%%%%%%%%%%%%%%%%%%%%%%%%%%%%

\end{document}